% !TeX root = ../paper_observer_consciousness.tex

Betrachtet man die Begriffe Beobachter und Bewusstsein aus einer 
allgemeinen Perspektive, so wirf nur einer dieser beiden 
tiefgreifende Fragen auf. Die Antwort auf die Frage, was ein Beobachter ist,
ist so grundlegend einfach wie beispielsweise die im
Duden gegebene Definition \Quote{jemand, der etwas oder jemanden beobachtet}\,\cite{duden_observer}.
Diese Einfachheit der Beschreibung eines Beobachters ist durch die Objektivität zu begründen,
die dem Begriff zu Grunde liegt. Im Gegensatz dazu ist das Bewusstsein etwas, das für jeden Menschen
nur subjektiv erfahrbar ist und obgleich es anzunehmen ist, dass auch alle anderen Menschen bewusst sind,
ist dies nicht ohne weiteres feststellbar. Eine umfassende Definitionen für Bewusstsein erweist sich daher als 
schwierig, sodass sich Definitionen häufig, auf einzelne Teilaspekte fokussieren, die mit dem Bewusstsein assoziiert 
werden.\,\cite{duden_consciousness}
Eine der angedeuteten tiefgreifenden Fragen, ist das von David Chalmers benannte \Quote{schwierige Problem}\,\cite{Chalmers_95}, welches unter Philosophen nicht unumstritten ist, jedoch im Grunde 
die Frage danach stellt, warum wir ein Bewusstsein haben. Ein Konzept, das diese Frage zu beantworten 
versucht, ist das des Dualismus. Dieses beschreibt neben der physischen Materie die Existenz einer nicht-physischen
\enquote{Lebenskraft} (Seele), die uns Menschen unser Bewusstsein verleiht.


Aus physikalischer Sicht ist schon die Definition eines Beobachters schwierig, da diese sich je nach 
Betätigungsfeld unterscheidet. Eine große Differenz zwischen den Bedeutungen eines Beobachters
ergibt sich beispielsweise, wie auch unter anderen Gesichtspunkten, 
zwischen der allgemeinen Relativitätstheorie und der Quantenmechanik. In erstere wird der Beobachte als ein
masse- und ausdehnungsloser Punkt angenommen, der folglich keine Auswirkung auf das Beobachtete hat.
In der quantenmechanischen Beschreibung kann argumentiert werden, dass ein Beobachter in irgendeiner 
Form an einem Messprozess beteiligt ist und somit einen Einfluss auf das System ausübt.
Die aktuelle Situation lässt sich mit den Worten von Max Tegmark wie folgt beschreiben: 
\begin{quote}
	\Quote{The only issue there is consensus on is that there is no
	consensus about how to define an observer and its role.}\,\cite{Tegmark_15_long}
\end{quote}   
Da schon der, allgemein betrachtet, noch eher einfache Begriff des Beobachters in der Physik bereits für Uneinigkeit sorgt,
ist anzunehmen, dass das Bewusstsein aus physikalischer Sicht mindestens genauso viele offene Fragen hinterlässt.
Zu einem gewissen Ausmaß ist dies Vermutung zutreffend, denn auf Fragen wie das \enquote{schwierige Problem} 
liefert die Physik bislang keine Antworten. Beispielsweise ist das Konzept des Dualismus von einem physikalischen 
Standpunkt aus nicht vertretbar, da eine Seele entweder keine Auswirkungen auf
die Elementarteilchen hat, aus denen unser Körper aufgebaut ist, und folglich nicht existiert oder aber doch Kräfte 
ausübt, wodurch die Seele wiederum physikalisch beschrieben werden kann. In beiden Fällen ist eine Unterscheidung von 
Materie und Seele somit nicht notwendig.
Eine Vereinfachung des Problems der Definition für Bewusstsein ergibt sich jedoch dadurch, dass das Bewusstsein in der Physik
allgemein unbeachtet bleibt, wodurch sich die Frage nach einer Definition gar nicht erst stellt.\,\cite{Tegmark_15_long} 




