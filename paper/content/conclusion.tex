% !TeX root = ../paper_observer_consciousness.tex

Wie schon im Allgemeinen ergeben sich auch in der Physik Problem eine Definition für Bewusstsein zu finden,
wobei in letzterer selbst eine allgemeine Definition für einen Beobachter fehlt. Es kann also vermutet werden,
dass eine mögliche einheitliche Definition für Beobachter das Bewusstsein in irgendeiner Weise einschließen 
könnte.

Bei der Auffassung des Beobachters als Teilsystem ergeben sich strikte Bedingungen an die Wechselwirkungsterme
mit Objekt und Umwelt, die dafür Sorge zu tragen haben, dass Korrelationen zu ausgewählten Eigenschaften der
Umgebung konstant bleiben. 
Ferner zeigte die Berechnung der Dekohärenz-Zeitskalen von Neuronen, dass diese und damit auch sämtliche Gehirnprozesse
als klassische zu beschreiben sind, da jegliche Superpositionen noch bei der Entstehung zerstört werden.

Die Betrachtung von notwendigen Eigenschaften bewusster Materie zeigt, dass obwohl Integrierte Information
und Unabhängigkeit sinnvolle Vertreter dieser Eigenschaften sind diese beiden allen noch nicht ausreichend
sind, um bewusste Materie vollständig zu beschreiben. 


