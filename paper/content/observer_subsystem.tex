% !TeX root = ../paper_observer_consciousness.tex

Ein theoretisches Konzept für die Beschreibung eines Beobachters,
ist die Einteilung des betrachteten Gesamtsystems, beschrieben durch 
den Hamiltonoperator $H$, in drei Teilsysteme der Form
\begin{empheq}{align}
	H &= H_{\mathrm{O}} + H_{\mathrm{S}} + H_{\mathrm{E}}\notag\\ 
	  &+ H_{\mathrm{OE}} + H_{\mathrm{SE}} + H_{\mathrm{OS}} + H_{\mathrm{OES}}.
\end{empheq}

Dabei werden die Teilsysteme so gewählt, dass das Objekt (O) alle Freiheitsgrade
enthält die für den Beobachter relevant sind, beispielsweise der Wert eines Messgeräts.
Genauso werden alle Freiheitsgrade des Beobachters, die dessen Wahrnehmung beschreiben 
in einem Teilsystem betrachtet, das Subjekt (S) genannt wird. Der Großteil aller 
Freiheitsgerade, der nach dieser Einteilung noch übrig ist und auch die Freiheitsgerade
aller Teilchen einschließt aus denen Beobachter und Messgerät bestehen, wird als Umgebung 
(E) beschrieben.

Durch die Einteilung in drei Teilsysteme ergibt sich im allgemeinen Fall ein komplexer Wechselwirkungsterm,
der aus den drei Wechselwirkungen von jeweils zwei der Teilsysteme miteinander und einem vierten Term besteht, der
die Wechselwirkung aller drei Teilsysteme beschreibt, für den jedoch im Folgenden $H_{\mathrm{OES}} = 0$
angenommen wird. 
Die drei übrigen Wechselwirkungsterme lassen sich mit bekannten Wechselwirkungen identifizieren. So beschreiben
$H_{\mathrm{OE}}$ und $H_{\mathrm{OE}}$ die Dekohärenz von dem Objekt respektive dem Subjekt durch die Wechselwirkung
mit der Umgebung. Der Term $H_{\mathrm{OS}}$ kann als Messung des Beobachters am System verstanden werden.

Damit ein solches Modell für den Beobachter (Subjekt) mit der Realität vereinbare Ergebnisse liefert,
müssen die Wechselwirkungen von Subjekt mit der Umgebung und dem Objekt eine hohes Maß an Korrelation
zu bestimmten Eigenschaften hervorrufen. Beispiele für diese Eigenschaften sind die Lichtintensität oder 
den Schalldruck, die wir mit unseren Sinnen relativ konstant wahrnehmen können. Ferner müssen auch Korrelationen
zu vergangenen Zuständen möglich sein, da wir als menschliche Beobachter die Fähigkeit haben uns an 
die Vergangenheit zu erinnern.

Um die Wirkung der einzelnen Wechselwirkungsterme zu verdeutlichen ist im Folgenden ein Beispiel
aus\,\cite{Tegmark_15_long} gegeben. Betrachtet wird hierbei 
%die reduzierte Dichtematrix $\rho_{\mathrm{OS}}$
ein Systems in denen sowohl das Objekt als auch das Subjekt nur einen Freiheitsgrad mit den jeweiligen 
Zuständen 
\begin{empheq}{align}
	\text{Subjekt}&: \ketsmiley, \ketneutrey, \ketfrowny\\
	\text{Objekt}&: \ketup, \ketdown
\end{empheq}
besitzen. Aus denen sich die 6 Basiszustände 
\begin{empheq}{equation}
	\ketsmileyup, \ketsmileydown,\ketneutreyup, \ketneutreydown,\ketfrownyup, \ketfrownydown
\end{empheq}
für das verbundene System aus ergeben. Für den Anfangszustand wird die Dichtematrix $\rho_{\mathrm{OS}}~=~  \ketneutreyup\braneutreyup$ angenommen. Der Term $H_{\mathrm{O}}$ beschreibt nun die Zeitentwicklung des
Objekts für die folgendes angenommen werden soll.
\begin{empheq}{equation}
	U\ketup = \exp(-iH_{\mathrm{O}}t)\ketup = \frac{1}{\sqrt{2}}\del{\ketup + \ketdown}
\end{empheq}
Diese Entwicklung erzeugt aus dem reinen Anfangszustand $\rho_{\mathrm{O}}~=~\ketup\braup$ einen
Zustand in Superposition
\begin{empheq}{equation}
U\rho_{\mathrm{O}}U^{\dagger} = \frac{1}{2}\del{\ketup\braup + \ketup\bradown + \ketdown\braup + \ketdown\bradown}.
\end{empheq}
Durch die Wechselwirkung mit der Umgebung findet nun (vollständige) Dekohärenz dieses Zustands statt, wodurch die 
Nebendiagonalelemente der Dichtematrix unterdrückt werden und man den Zustand
\begin{empheq}{equation}
 \rho^{\prime}_{\mathrm{O}} = \frac{1}{2}\del{\ketup\braup + \ketdown\bradown}
\end{empheq}
erhält. 
Die Messung des Subjekts am Objekt sei in der Art, dass die neutrale Stimmung $\ketneutrey$ des Beobachters bei Beobachtung eines $\ketup$ zu $\ketsmiley$ und bei Beobachtung  von $\ketdown$ zu $\ketfrowny$ übergeht.
Formal lassen sich diese Annahmen mit $U\ketneutreyup = \ketsmileyup$ und $U\ketneutreydown = \ketfrownydown$ beschreiben, wobei $U$ die Zeitentwicklung in Abhängigkeit von $H_{\mathrm{OS}}$ darstellt.
Bei der Beobachtung des Zustands  $\rho^{\prime}_{\mathrm{O}}$ durch einen anfangs neutralen Beobachter ergibt sich 
entsprechend
\begin{empheq}{align}
	 \rho^{\prime}_{\mathrm{OS}}&= \frac{1}{2} U\del{\ketup\braup + \ketdown\bradown} \otimes \ketneutrey\braneutrey U^{\dagger}\notag\\
	 &= \frac{1}{2}(\ketsmileyup\brasmileyup + \ketfrownydown\brafrownydown).
\end{empheq}

Analog wirken die interne Dynamik des Subjekts $H_{\mathrm{S}}$  und die Wechselwirkung des Systems mit der
Umgebung $H_{\mathrm{SE}}$ , wodurch zunächst Superpositionen erzeugt und durch Dekohärenz wieder unterdrückt werden.
Aus dem System des neutralen Beobachters $\rho_{\mathrm{S}} = \ketneutrey\braneutrey$ ergibt sich durch Zeitentwicklung $U$
\begin{empheq}{equation}
\label{eq:example_Hs}
U\rho_{\mathrm{S}}U^{\dagger} =\frac{1}{2}\del{\ketsmiley\brasmiley + \ketsmiley\brafrowny + \ketfrowny\brasmiley + \ketfrowny\brafrowny}.
\end{empheq}
Und nach Wirkung der  Dekohärenz
\begin{empheq}{equation}
\rho^{\prime}_{\mathrm{S}} =\frac{1}{2}\del{\ketsmiley\brasmiley + \ketfrowny\brafrowny}.
\end{empheq}

Für den Fall eines menschlichen Beobachters lässt sich nun die Frage stellen, ob die Superpositionen durch die 
Zeitentwickung überhaupt entstehen oder aber die Dekohärenz auf so kleinen Zeitskalen abläuft, dass diese 
förmlich  instantan zerstört werden. Diese Frage wird im folgenden Abschnitt unter Betrachtung von Neuronen im 
menschlichen Gehirn beantwortet.   