% !TeX root = ../paper_observer_consciousness.tex

Die bekannten Aggregatzustände lassen sich durch die Unterschiede in ihren Eigenschaften einteilen,
so haben Festkörper eine quasi-unendliche Viskosität. Alle übrigen Stoffe zeigen weiter Unterschiede,
beispielsweise sind Flüssigkeiten weniger kompressible und Gase und Plasmen lassen sich durch ihr
Leitfähigkeit unterscheiden. 

In analoger Form lässt sich nun die Frage nach den Eigenschaften stellen, die Materie haben muss, 
die bewusst ist. Ähnliche Überlegungen wurden bereits angestellt, um festzustellen welche Eigenschaften 
\emph{computronium} haben muss, ein Zustand von Materie der \enquote{berechnen} kann.
Zwei mögliche Eigenschaften von bewusster Materie, Integrierte Information und Unabhängigkeit werden im 
folgenden erläutert.
