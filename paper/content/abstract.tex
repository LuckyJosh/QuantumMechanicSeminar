% !TeX root = ../paper_observer_consciousness.tex
\vspace*{0.75cm}
\begin{center}
	\parbox{0.8\textwidth}{Im Folgenden sollen die Grundzüge der physikalischen Beschreibung von Beobachtern und
		dem Bewusstsein betrachtet werden. Dabei wird zunächst auf die Probleme der 
		Definitionen eingegangen. Darauffolgend werden theoretische Modelle betrachtet, die
		Möglichkeiten zur Beschreibung eines Beobachters und der Eigenschaften 
		von bewusster Materie bieten. Ferner wird durch Berechnung der Dekohärenz-Zeitskalen 
		von Gehirnprozessen argumentiert, dass diese als klassisch zu behandeln sind.}
\end{center}
\vspace{0.75cm}