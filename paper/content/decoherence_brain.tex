% !TeX root = ../paper_observer_consciousness.tex

Ein Gehirnprozess in dem quantenmechanische Superpositionen denkbar sind
die Aktivität von Neuronen, den Grundbausteinen der Informationsverarbeitung im 
Gehirn. Das komplexe Netzwerk aus $\sim \num{e11}$ Neuronen kann auch durch
Forschungsergebnisse anderer Wissenschaften mit dem Bewusstsein in Zusammenhang
gebracht werden. 

Eine schematische Darstellung einer Nervenzelle ist in \cref{fig:neuron} gezeigt.
Auf das, für diese Betrachtung, Notwendigste reduziert stellen Neuronen im Prinzip
ein System mit zwei möglichen Zuständen dar, die feuern (aktiv) und 
nicht-feuern (passiv) genannt werden. Der Zustand ist dabei von der Konzentration
von Na$^{+}$-Ionen im Inneren der Nervenzelle abhängig, bei hohen Konzentrationen
feuert die Nervenzelle bei niedrigen nicht. Eine Superposition zwischen einer 
feuernden und nicht-feuernden Nervenzelle lässt sich demnach vereinfacht als räumliche 
Superposition von Na$^{+}$-Ionen annehmen. Die räumliche Trennung dieser beiden Orte 
ist dabei von der Größenordnung der Zellwände $h \sim \SI{10}{\nano\meter}$.

Um nun Aussagen über die Dekohärenz-Zeitskalen machen zu können muss zunächst 
die Anzahl der Ionen bestimmt werden, die sich gleichzeitig in Superposition befinden.
Diese Anzahl wird über eine einfache Abschätzung der Ladungsverteilung in der Nervenzelle
ermittelt. Im feuernden Zustand beträgt die Potentialdifferenz zwischen dem Inneren und Äußeren
der Zelle $U_1 \sim \SI{0.03}{\volt}$ während diese im nicht-feuernden Zustand $U_0 \sim \SI{-0.07}{\volt}$
beträgt. Die Anzahl $N$ der Na$^{+}$-Ionen ergibt sich damit und realistischen Abschätzungen für 
die Ausmaße einer Nervenzelle zu 
\begin{empheq}{equation}
		N = \frac{\pi dLf\epsilon_{0}}{qh}\del{U_{1} -U_{0}}
\end{empheq}
