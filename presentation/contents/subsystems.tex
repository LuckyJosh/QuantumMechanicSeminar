\begin{frame}{Beobachter als Teilsystem}
	\begin{itemize}
		\item{Zerlegung eines Systems beschreiben durch $H$ und $\rho$}
	\end{itemize}
	\begin{beamerboxesrounded}{3 Teilsysteme + Wechselwirkung}
		\begin{empheq}{align*}
		H &= H_{\mathrm{O}} + H_{\mathrm{E}} + H_{\mathrm{S}} + H_{\mathrm{int}}\\
		H_{\mathrm{int}} &= H_{\mathrm{OE}} + H_{\mathrm{ES}} + H_{\mathrm{OS}} + H_{\mathrm{OES}}
		\end{empheq}
		\vspace{-0.5cm}
	\end{beamerboxesrounded} 
	\begin{itemize}
		\item{Subjekt (S): Freiheitsgrade der subjektiven Wahrnehmung des Beobachters}
		\item{Objekt (O): Zu beobachtende Freiheitsgrade}
		\item{Umgebung (E): Alle übrigen Freiheitsgrade des Systems}
	\end{itemize}
\end{frame}
