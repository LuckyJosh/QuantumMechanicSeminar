\begin{frame}{Autonomie}
		\only<1>{\hspace{-0.24cm}\includegraphics[scale=0.3]{graphics/property_principles_3.jpg}\,\cite{Tegmark_15_long}}
		\only<2>{\hspace{-0.32cm}\includegraphics[scale=0.3]{graphics/property_principles_4.jpg}\,\cite{Tegmark_15_long}}
		\alt<3>{\begin{itemize}
				\item{Vereinigung von \textcolor{vertexLightGrey}{Dynamik} und Unabhängigkeit}
				\begin{itemize}
					\item{\textcolor{vertexLightGrey}{Informationsverarbeitung}}
				\end{itemize}
				\item{Maß für Dynamik}
			\end{itemize}
			\begin{beamerboxesrounded}{Energie-Kohärenz}
				\begin{empheq}{equation*}
					\delta H = \frac{1}{\sqrt{2}} \norm{\dot{\rho}} = \sqrt{\Tr\sbr{H^{2}\rho^{2} - H\rho H\rho}}
				\end{empheq}
				\vspace{-0.5cm}
			\end{beamerboxesrounded}}{}
			\alt<4>{\begin{itemize}
					\item{Unterschiedliche Grade an Dynamik:}
				\end{itemize}
				\begin{columns}
					\begin{column}{0.3\textwidth}
						\centering
						\includegraphics[scale=.3]{graphics/autonomy_static.jpg}
					\end{column}
					\begin{column}{0.3\textwidth}
						\centering
						\includegraphics[scale=.3]{graphics/autonomy_chaotic.jpg}
					\end{column}
					\begin{column}{0.3\textwidth}
						\centering
						\includegraphics[scale=.3]{graphics/autonomy_simple.jpg}\,\cite{Tegmark_15_long}
					\end{column}
				\end{columns}
				\begin{itemize}
					\item{Reduktion der maximalen Energie-Kohärenz um wenige Prozent}
					\begin{itemize}
						\item{Komplexe, chaotische Dynamik möglich}
					\end{itemize}
				\end{itemize}}{}
\end{frame}