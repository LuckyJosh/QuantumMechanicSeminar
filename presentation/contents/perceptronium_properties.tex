\begin{frame}{Eigenschaften von Perceptronium}
	\begin{itemize}
		\item{Aktive Forschung z.B. in der Neurowissenschaft}
		\begin{itemize}
			\item{G. Tononi (\textsc{Integrated Information Theory})\,\cite{Tononi_08}}
  		\end{itemize}
	\end{itemize}
	\begin{beamerboxesrounded}{Integrierte Information $\textcolor{white}{\Phi}$ (abgewandelt)}
		\begin{empheq}{equation*}
			\Phi = I_{\mathrm{min}} = \min_{\rho_1,\rho_2} \del{S(\rho_{1}) + S(\rho_{2}) - S(\rho)}
		\end{empheq}
		\vspace{-0.5cm}
		\begin{empheq}{equation*}
			\small I: \text{Transinformation}, S = - \Tr[\rho\log_{2}(\rho)]
		\end{empheq}
		\vspace{-0.5cm}
	\end{beamerboxesrounded}
	\begin{itemize}
		\item{Minimale Transinformation nach einem Schnitt der das System in zwei teilt}
			\begin{itemize}
				\item{\Quote{the cruelest cut} - Giulio Tononi}
				\item{Maximale Unabhängigkeit der Teilsysteme,\\ $\Phi = 0 \Leftrightarrow$ vollständig unabhängig}
			\end{itemize}
	\end{itemize}
\end{frame}