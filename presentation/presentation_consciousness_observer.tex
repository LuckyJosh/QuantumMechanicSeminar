%%%%%%%%%%%%%%%%%%%%%%% Grundeinstellungen %%%%%%%%%%%%%%%%%%%%%%%%%%%
% 'Artikel' Dokumentenklasse und Standardschriftgröße  
\documentclass[11pt,landscape]{beamer}


% Setzt das Papierformat und den Rand auf 2.5cm                                                      
%\usepackage[paper=a4paper,left=2.5cm,right=2.5cm,top=2.5cm,bottom=2.5cm]{geometry}  

% Setzt die Einrückung von Absätzen auf gegebenen Abstand
\setlength{\parindent}{0mm}

% Legt Zeilenabstand fest                                                         
%\usepackage[onehalfspacing]{setspace}                                               

% Legt FontKodierung fest
\usepackage[T1]{fontenc}

% Legt Zeichenkodierung fest                                                            
\usepackage[utf8]{inputenc}                                                         

% Neue Deutsche Rechtschreibung
\usepackage[ngerman]{babel}  

% Versieht Referenzen mit Bezeichnung des Objektes                                                      
%\usepackage[ngerman]{varioref}   

% Empfohlener T1-Font für deutsche Texte                                               
\usepackage{lmodern} 

% Deutsche Zitate mit \enqoute{},\enqoute*{} 
\usepackage[babel,german=quotes]{csquotes}

%Einstellungen für Bibliographien mit "biber"                                                              
\usepackage[backend=biber, style=numeric-verb,sorting=none]{biblatex}

%%%%%%%%%%%%%%%%%%%% Seitenlayout %%%%%%%%%%%%%%%%%%%%%%%%%%%%%%%%%%%%

% Ermöglicht detailierte Bearbeitung der Kopf- und Fußzeile
\usepackage{fancyhdr} 

% Setzt Kopf und Fußzeile zurück                                                              
%\fancyhf{} 	     
         
% Höhe der Kopfzeile                                                          
%\setlength{\headheight}{28.0pt}   

% Höhe der Fußzeile                                                  
%\setlength{\footskip}{18.0pt}                                                      

% Dicke des Kopfzeilentrennstrichs
%\renewcommand{\headrulewidth}{.5 pt}     

% Dicke des Fußzeilentrennstrichs                                     
%\renewcommand{\footrulewidth}{.5 pt}                                                 

%Test ob Variable gesetzt ist oder nicht
%\ifdefined\ExperimentTitle% 
	% Angabe Links-Oben
%	\lhead{\textbf{\ExperimentTitle}}
%\else%
%    \lhead{\textbf{\textcolor{red}{VERSUCHNAME!!!}}}
%\fi%      
                                                            
% Angabe Mitte-Oben
%\chead{}  

% Angabe Rechts-Oben                                                                         
%\rhead{\today}  

% Angabe Links-Unten                                                                    
%\lfoot{}        

% Angabe Mitte-Unten                                                                   
%\cfoot{\textbf{\thepage\ von \pageref{LastPage}}}  

% Angabe Rechts-Unten                                 
%\rfoot{}                                           

% Anwenden des erweiterten Seitenlayouts
%\pagestyle{fancy}      
%%%%%%%%%%%%%%%%%%%%%%% Referenzen %%%%%%%%%%%%%%%%%%%%%%%%%%%%%%%%%%%

% Macht die letzte Seitenzahl referenzierbar mit \pageref{Lastpage}
\usepackage{lastpage}

% Ermöglicht Referenzen im Dokument                                                               
\usepackage{hyperref}
	%Keine Hervorhebung der Referenzen
	\hypersetup{hidelinks}

%Verbesserte Referenzen
\usepackage[german]{cleveref}
                                                              
%%%%%%%%%%%%%%%%%%%%%%%% MINT %%%%%%%%%%%%%%%%%%%%%%%%%%%%%%%%%%%%%%%%
% Fügt mathematische Symbole hinzu, setzt Grenzen, Limiten und Indizes unter das Symbol und nicht dahinter
%\usepackage[sumlimits,intlimits,namelimits]{amsmath}  
\usepackage{amsmath}  
% Fügt Symbole wie z.B. Zahlenmengen wie $\mathbb{R}$ hinzu                             
\usepackage{amssymb}    
                                                             
% Beweißumgebung
\usepackage{amsthm}  

% Font für Mathematikumgebung                                                              
\usepackage{amsfonts}    
                                                          
% Verbesserte Gleichungsumgebung mit \begin{empheq}[<Aussehen>]{<Umgebungstyp>} ... \end{empheq}
\usepackage{empheq}      

% Ergänzungen für physikalische Arbeiten
\usepackage{physics}

% Chemische Struktur und Summenformeln mit \ce{<Summenformel>}                                                          
%\usepackage[version=3]{mhchem}  

% Chemische Valenzstrichformeln für ganze Moleküle mit\chemfig{<Molekül-Aufbau>}                                                   
%\usepackage{chemfig}                                                               

% Verbesserte Formatierung von größen mit Einheiten 
\usepackage[binary-units=true]{siunitx}                                                               
	% 'Mal'-Zeichen auf \cdot und Dezimaltrennzeichen auf ',' 
	\sisetup{locale = DE,prefixes-as-symbols = true}                                   
                                                                            
	% Vereinfachtes eintragen von Unsicherheiten mit '42.6(4)' --> '42.6 +/- 0.4'          
	\sisetup{separate-uncertainty = true}                                                                                                                    

 
                                                            
% Fügt verbesserte Vektorpfeile hinzu \vv{<Vektorname>} 
\usepackage[b]{esvect}  

% Brüche mit "/" im Text mit \sfrac{}                                                            
\usepackage{xfrac}

% Darstellung von 2D Feldern 
\usepackage{array}

% Differentialoperatoren,-quotienten und Klammern mit \od[]{}{}, \pd[]{}{}, \del{}, \sbr{}, \cbr{}
\usepackage{commath}

% Pseudocode-Umgebungen 																
\usepackage{algorithmicx}
\usepackage{algpseudocode}

% Einbinden von SourceCode Dateien (listing)
\usepackage{listings}
	% Auswählen der Programmiersprache
	\lstset{language=Python}

%%%%%%%%%%%%%%%%% Seiten- und Floateinstellungen %%%%%%%%%%%%%%%%%%%%%
% Einbinden von Grafiken mit '\includeudegraphics[<Optionen>]{<Grafikpfad>}' und Veränderungen im Text, wie z.B. Schriftfarbe 
\usepackage{graphicx}
\usepackage{xcolor} 
\xdefinecolor{tugreen}{RGB}{128, 186, 38}

% Fügt Möglichkeit für textumflossende Grafiken und Tabellen hinzu \begin{floating<figure/table>}[option]{width} ... \caption ... \end{floatingfigure}                                                              
\usepackage{floatflt}        
\usepackage{wrapfig}

% Ermöglicht das Hinzufügen von Unterabbildung zu einer Abbildung                                                 
%\usepackage{subfig} 

% Verhindert das Wandern von "floating" Umgebungen über eine Bestimmte Grenze (hier: Sections) oder mit \FloatBarrier                                                             
\usepackage[section]{placeins}

% Ermöglicht Zeichnungen im Dokument \begin{tikzpicture} ... \end{tikzpicture}
\usepackage{tikz}  
	% Fügt zusätzlichen Pfeilspitzen hinzu                                                                 
	\usetikzlibrary{arrows}                                                         
	\usetikzlibrary{calc}    
% Besseres Tabellenlayout                                                        
\usepackage{booktabs}

% Skalierbare Umgebung für Tabellen und Bilder
\usepackage[export]{adjustbox}

% Bearbeiten von Bild-/Tabellenunterschriften
\usepackage[font=small,labelfont=bf]{caption}

% Seiten im Querformat mit \begin{landscape}...\end{landscape} 
\usepackage{pdflscape}

% Ermöglicht detailierte Einstellungen an Aufzählungssymbolen
%\usepackage{enumitem} 

% Ermöglicht das Einfügen mehrerer Einträge in eine Tabellenzelle, getrennt von einem '\'
% \backslashbox{<Eintrag unten-links>}{<Eintrag oben-rechts>} TIPP: Leerzeichen                                                          
%\usepackage{slashbox}   

%Einbinden von Textdateien
\usepackage{fancyvrb}
% redefine \VerbatimInput
\RecustomVerbatimCommand{\VerbatimInput}{VerbatimInput}%
{fontsize=\footnotesize,
	%
	frame=lines,  % top and bottom rule only
	framesep=1em, % separation between frame and text
	rulecolor=\color{gray},
	%
	label=\fbox{\color{black} Kenngrößen},
	labelposition=topline,
	%
	commandchars=\|\{\}, % escape character and argument delimiters for
	% commands within the verbatim
	commentchar=\#       % comment character
	}
	
%%%%%%%%%%%%%%%%%%%%%%%%%%%%%%%%%%%%%%%%%%%%%%%%%%%%%%%%%%%%%%%%%%%%%%

% Fügt verbesserte Unterschtreichungen hinzu, z.B. doppelt, gezackt, gewellt, etc. mit \uline{},\uuline{},
\usepackage[normalem]{ulem}                                                                                                               

%Verbesserte Verwendung von Daten
\usepackage{scrdate}

% Zusaätzliche Symbole
\usepackage{pifont}

% Fügt extra Symbole hinzu 
\usepackage{textcomp}  

% Emojis
%\usepackage{styles/coloremoji}
%\Smiley,\Sadey,\Neutrey
\usepackage{tikzsymbols}
\newcommand{\ketsmiley}{\ensuremath{\ket{\Smiley}}}
\newcommand{\brasmiley}{\ensuremath{\bra{\Smiley}}}
\newcommand{\ketneutrey}{\ensuremath{\ket{\Neutrey}}}
\newcommand{\braneutrey}{\ensuremath{\bra{\Neutrey}}}
\newcommand{\ketfrowny}{\ensuremath{\ket{\Sadey}}}
\newcommand{\brafrowny}{\ensuremath{\bra{\Sadey}}}
\newcommand{\ketup}{\ensuremath{\ket{\uparrow}}}
\newcommand{\braup}{\ensuremath{\bra{\uparrow}}}
\newcommand{\ketdown}{\ensuremath{\ket{\downarrow}}}
\newcommand{\bradown}{\ensuremath{\bra{\downarrow}}}

\newcommand{\ketsmileyup}{\ensuremath{\ket{\Smiley\uparrow}}}
\newcommand{\ketsmileydown}{\ensuremath{\ket{\Smiley\downarrow}}}
\newcommand{\brasmileyup}{\ensuremath{\bra{\Smiley\uparrow}}}
\newcommand{\brasmileydown}{\ensuremath{\bra{\Smiley\downarrow}}}
\newcommand{\ketneutreyup}{\ensuremath{\ket{\Neutrey\uparrow}}}
\newcommand{\ketneutreydown}{\ensuremath{\ket{\Neutrey\downarrow}}}
\newcommand{\braneutreyup}{\ensuremath{\bra{\Neutrey\uparrow}}}
\newcommand{\braneutreydown}{\ensuremath{\bra{\Neutrey\downarrow}}}
\newcommand{\ketfrownyup}{\ensuremath{\ket{\Sadey\uparrow}}}
\newcommand{\ketfrownydown}{\ensuremath{\ket{\Sadey\downarrow}}}
\newcommand{\brafrownyup}{\ensuremath{\bra{\Sadey\uparrow}}}
\newcommand{\brafrownydown}{\ensuremath{\bra{\Sadey\downarrow}}}
%%%%%%%%%%%%%%%%%%%%%%%%%%%%%%%%%%%%%%%%%%%%%%%%%%%%%%%%%%%%%%%%%%%%%%

\title{} 
\author{} 
% Literaturfile
\addbibresource{sources.bib}
%%%%%%%%%%%%%%%%%%%%Aänderungen & Eigene Befehle%%%%%%%%%%%%%%%%%%%%%%                                                     
% Abstand zwischen Text und Fußnoten
\setlength{\skip\footins}{2cm}  

% Abstand zwischen Fußnoten                                                    
%\setlength{\footnotesep}{2cm}

% Abstand zwischen \items 
%\setlength{\itemsep}{7.5pt}

% Änderung der Fußnotenmarkierungen (hier: Zahlen in Kreisen)																    % Fußnoten mit Zahlen in Kreisen
\renewcommand\thefootnote{\ding{\numexpr171+\value{footnote}}}


\renewcommand{\i}{\ensuremath{\textsl{i}}}
\newcommand{\e}{\ensuremath{\textsl{e}}}
\renewcommand{\Im}{\mathrm{Im}\,}
\renewcommand{\Re}{\mathrm{Re}\,}


% Funktionsmakros mit größenvariablen Klammern benötigt \usepackage{commath}
\newcommand{\E}[1]{\e^{#1}}
\newcommand{\Exp}[1]{\exp\!\del{#1}}
\newcommand{\Ln}[1]{\ln\!\del{#1}}
\newcommand{\Log}[2][10]{\log_{#1}\!\del{#2}}
\newcommand{\Sin}[2][]{\sin^{#1}\!\del{#2}}
\newcommand{\Cos}[2][]{\cos^{#1}\!\del{#2}}
\newcommand{\Tan}[2][]{\tan^{#1}\!\del{#2}}
\newcommand{\Sinh}[1]{\sinh\!\del{#1}}
\newcommand{\Cosh}[1]{\cosh\!\del{#1}}
\newcommand{\Tanh}[1]{\tanh\!\del{#1}}
\newcommand{\Arcsin}[1]{\arcsin\!\del{#1}}
\newcommand{\Arccos}[1]{\arccos\!\del{#1}}
\newcommand{\Arctan}[1]{\arctan\!\del{#1}}
\newcommand{\Arsinh}[1]{\arsinh\!\del{#1}}
\newcommand{\Arcosh}[1]{\arcosh\!\del{#1}}
\newcommand{\Artanh}[1]{\artanh\!\del{#1}}

\newcommand{\Cov}[2]{\text{cov}\!\del{#1,#2}}
\newcommand{\Erw}{\mathrm{E}}


%\DeclarePairedDelimiter{\abs}{\lvert}{\rvert}
\DeclarePairedDelimiter{\mean}{\langle}{\rangle}

%%%%%%%%%%%%%%%%%%%%%%%%%%%%%%%%%%%%%%%%%%%%%%%%%%%%%%%%%%%%%%%%%%%%%%%%%%
\mode<presentation>
{	
	\usepackage{styles/beamerthemevertex}
	%\usetheme{styles/TUDortmund}
}
% Include intermediate TOCs automatically.
% Completely pointless for these slides, however
%\AtBeginSection[]
%{
%  \begin{frame}<beamer>
%    \frametitle{\"Ubersicht}
%    \tableofcontents[currentsection]
%  \end{frame}
%}
%\AtBeginSubsection[]
%{
%	\begin{frame}<beamer>
%		\frametitle{\"Ubersicht}
%		\tableofcontents[currentsubsection]
%	\end{frame}
%}


%
% Options for navigation symbols
%
% a) Small set of navigation symbols
%\setbeamertemplate{navigation symbols}{
%  \insertbackfindforwardnavigationsymbol
%  \hspace{0.5em}
%  \insertslidenavigationsymbol
%  \insertframenavigationsymbol
%  \insertsubsectionnavigationsymbol
%  \insertsectionnavigationsymbol
%  \insertdocnavigationsymbol
%  \hspace*{\textwidth}\hspace*{0.2cm}
%}
%
% b) Suppress navigation symbols
\setbeamertemplate{navigation symbols}{}

\newcommand{\lc}[1]{\texttt{\textbackslash#1}}
\newcommand{\lcp}[2]{\texttt{\textbackslash#1\{#2\}}}
\newcommand{\lco}[2]{\texttt{\textbackslash#1[#2]}}
\newcommand{\lcop}[3]{\texttt{\textbackslash#1[#2]\{#3\}}}
\newcommand{\lcpp}[3]{\texttt{\textbackslash#1\{{#2}\}\{{#3}\}}}
\newcommand{\lcppp}[4]{\texttt{\textbackslash#1\{#2\}\{#3\}\{#4\}}}
\newcommand{\lcopp}[4]{\texttt{\textbackslash#1[#2]\{#3\}\{#4\}}}
\newcommand{\lenv}[2]{\texttt{\textbackslash{}begin\{#1\}\linebreak[0]#2\linebreak[0]\textbackslash{}end\{#1\}}}

\newcommand{\todo}[2]{\colorbox{orange}{\textbf{#1:} {#2}}}
\newcommand{\ds}{\ensuremath\displaystyle}
\newenvironment{determinante}[1]{%
	\ensuremath\left|\begin{array}{#1}}{%
	\end{array}\right|}

%%%%%%%%%%%%%%%%%%%%%%%%%%%%%%%%%%%%%%%%%%%%%%%%%%%%%%%%%%%%%%%%%%%%%%%%%%
\usepackage{multimedia}
\usepackage{appendixnumberbeamer}

\newcommand{\separatorslide}{%
\begin{frame}[plain]%
	\vspace*{1.5cm}%
	\begin{beamercolorbox}[wd=\paperwidth,ht=0.33\textheight,dp=2em]{title page}%
		\begin{minipage}{0.99\paperwidth}%
			\begin{flushright}%
				\textbf{\Huge\thesection. \insertsection}\par%
				\ifx\insertsubsection\@empty%
				%\vspace*{2.0cm}%
			     \else%
			     \huge
			     \insertsubsection
				 \vspace*{1cm}%
				 \fi%
			\end{flushright}%
		\end{minipage}%
	\end{beamercolorbox}%
\end{frame}%
\addtocounter{framenumber}{-1}}%






\newcommand{\Quote}[1]{\enquote{\emph{#1}}}
\newcommand{\altemph}[2]{\alt<#2>{\bfseries#1}{#1}}
%\newcommand{\ac}[2]{\textcolor{}}

\linespread{1.1}
%\parskip{2em}

%\setbeameroption{show notes} %un-comment to see the notes

\makeatletter
\def\beamer@framenotesbegin{% at beginning of slide
	\gdef\beamer@noteitems{}%
	\gdef\beamer@notes{{}}% used to be totally empty.
}
\makeatother



\title{Beobachter und Bewusstsein}
%\subtitle{A clean, basic beamer theme}
\date{\today}
\author{Joshua Luckey}
\institute{TU Dortmund}

\begin{document}
	
	\maketitle
	
	%\begin{frame}
%	\large
%	\begin{quote}
%%		\enquote{Sicherheit ist relativ, wie alles Irdische: Was Menschen schaffen,
%%		kann von Menschen zerlegt, geknackt oder vernichtet werden.}
%		\enquote{\,Alles, was schiefgehen kann, wird auch schiefgehen.}
%		\begin{flushright}
%%			 \upshape-Harald Gebert\cite{QuoteGebert}
%			 \upshape-Edward A. Murphy\,\cite{QuoteMurphy}
%		\end{flushright}
%	\end{quote}
\end{frame}
	
	\begin{frame}{Inhalt}
		\tableofcontents%[hideallsubsections]
	\end{frame}
	
	
	\section{Was sind Beobachter und Bewusstsein?}
		\separatorslide
		\begin{frame}{Beobachter}
	\begin{itemize}
		   \item{Duden: \Quote{jemand, der etwas oder jemanden beobachtet}\,\cite{duden_observer}}
		   \item{Allgemein: Objektiv, einfach zu verstehen; Passivität}
	\end{itemize}
	\vspace{.5cm}
	\begin{columns}
		\alt<1>{\begin{column}{0.45\textwidth}
			\centering 
			\includegraphics[scale=0.3]{graphics/em_frankreich_island.jpg}\cite{pic_football}
		\end{column}
		\begin{column}{0.45\textwidth}
			\centering 
			\includegraphics[scale=0.69]{graphics/Astronomer.jpg}\cite{pic_astronomer}	
		\end{column}}{}
		
		\alt<2>{\begin{column}{\textwidth}
			\centering 
			\includegraphics[scale=0.145]{graphics/Vortrag_PMI.jpg}\cite{pic_presentation}
		\end{column}}{}
	\end{columns}
\end{frame}
		\begin{frame}{Bewusstsein}
	\only<1>{Definition Duden\,\cite{duden_consciousness}:
	\begin{itemize}
		   \item{\Quote{Zustand, in dem man sich einer Sache bewusst ist; deutliches Wissen von etwas, Gewissheit}}
		   \item{\Quote{Gesamtheit der Überzeugungen eines Menschen, die von ihm bewusst vertreten werden}}
		    \item{\Quote{(Psychologie) Gesamtheit aller jener psychischen Vorgänge, durch die sich der Mensch der Außenwelt und seiner selbst bewusst wird}}
		   \item{\Quote{Zustand geistiger Klarheit; volle Herrschaft über seine Sinne}}
	\end{itemize}}
	
	\only<2->{\begin{itemize}
			\item{Duden: \Quote{(Psychologie) Gesamtheit aller jener psychischen Vorgänge, durch die sich der Mensch der Außenwelt und seiner selbst bewusst wird}\,\cite{duden_consciousness}}
			\item{Allgemein: Subjektiv, nicht fassbar nur erfahrbar}
			\item{\emph{Das schwierige Problem}: Warum nehmen wir wahr?}
			\begin{quote}
				\enquote{What is your extra ingredient, and why should that account for conscious experience?}
				- David Chalmers\,\cite{Chalmers_95} 
			\end{quote}
			\item{Häufige Antwort: Dualismus}
				\begin{itemize}
					\item{physische Materie} 
					\item{nicht-physische \enquote{Lebenskraft} (Seele)}
				\end{itemize}
	\end{itemize}}
\end{frame}
	
	\section{Betrachtung und Probleme in der Physik}
		\separatorslide
		\begin{frame}{Beobachter}
	\begin{itemize}
		\item{Definition abhängig von Betätigungsfeld}
		\item{Allgemeine Relativitätstheorie:}
		\begin{itemize}
			\item{keine Masse oder Ausdehnung}
			\item{keinen Einfluss auf das Beobachtete}
		\end{itemize}
		\item{Quantenmechanik:}
		\begin{itemize}
			\item{Einfluss: Kollaps der Wellenfunktion?}
		\end{itemize}
		\item{\Quote{The only issue there is consensus on is that there is no
				consensus about how to define an observer and its role.}\\ - Max Tegmark\,\cite{Tegmark_15_long}}
	\end{itemize}
\end{frame}
		\begin{frame}{Bewusstsein}
	\begin{itemize}
		\item{im allgemeinen unbeachtet}
		\begin{itemize}
			\item{\Quote{ An other argument physics has been managed just fine for hundreds of years avoiding this subject and should therefore keep doing so.} - Max Tegmark\,\cite{Tegmark_15_long}}
%			\item{\Quote{A commonly held view is that consciousness is irrelevant to physics and should therefore not be discussed in physics papers.}\\ - Max Tegmark\,\cite{Tegmark_15_long}}
		\end{itemize}
		\item{keine Lösung für das \emph{schwierige Problem}}
		\begin{itemize}
			\item{Dualismus nur schwer zu vertreten}
			\end{itemize}
		\item{Einfluss von respektive auf Quantenmechanik unklar}
		\begin{itemize}
			\item{Gehirn: nass und warm }
		\end{itemize}
	\end{itemize}
\end{frame}
		
	\section{Beobachter als Teilsystem}
		\separatorslide
		\begin{frame}{Beobachter als Teilsystem}
	\begin{itemize}
		\item{Zerlegung eines Systems beschreiben durch $H$ und $\rho$}
	\end{itemize}
	\begin{beamerboxesrounded}{3 Teilsysteme + Wechselwirkung}
		\begin{empheq}{align*}
		H &= H_{\mathrm{O}} + H_{\mathrm{E}} + H_{\mathrm{S}} + H_{\mathrm{int}}\\
		H_{\mathrm{int}} &= H_{\mathrm{OE}} + H_{\mathrm{ES}} + H_{\mathrm{OS}} + H_{\mathrm{OES}}
		\end{empheq}
		\vspace{-0.5cm}
	\end{beamerboxesrounded} 
	\begin{itemize}
		\item{Subjekt (S): Freiheitsgrade der subjektiven Wahrnehmung des Beobachters}
		\item{Objekt (O): Zu beobachtende Freiheitsgrade}
		\item{Umgebung (E): Alle übrigen Freiheitsgrade des Systems}
	\end{itemize}
\end{frame}

		\begin{frame}{Implikationen dieses Modells}
	\begin{itemize}
		\item{Prämisse: Freiheitsgrade des Subjekts sind die Wahrnehmungen des Beobachters}
		\item{Hohe Korrelation zu einer Auswahl von Eigenschaften der Umgebung und des Objekts}
		\begin{itemize}
			\item{Aufnahme von Reizen durch Sinnesorgane}
			\item{Korrelation zu vergangenen Zuständen}
		\end{itemize}
		\item{Transinformation zwischen Subjekt und Objekt + Umgebung relativ konstant}
			\begin{itemize}
					\item{Information über Umwelt durch Sinne}
					\item{Zunahme durch Lernen, Abnahme durch Vergessen}
			\end{itemize}
	\end{itemize}
\end{frame}
		\begin{frame}{Beispiel: $H_{\mathrm{O}}$, $H_{\mathrm{OE}}$, $H_{\mathrm{OS}}$}
	\alt<1>{\begin{itemize}
		\item{Betrachtung mit je einem Freiheitsgrad für (S) und (O)}
		\begin{itemize}
			\item{Subjekt: \ketsmiley, \ketneutrey, \ketfrowny}
			\item{Objekt: \ketup, \ketdown}
		\end{itemize}
		\item{Gesamtsystem S $\otimes$ O mit 6 Basiszuständen:\\
			\ketsmileyup, \ketsmileydown,\ketneutreyup, \ketneutreydown,\ketfrownyup, \ketfrownydown}
		\item{Dichtematrix $\rho = \ketneutreyup\braneutreyup$ als Anfangszustand:\\}
			\vspace{0.2cm}
			\centering
			\includegraphics[scale=0.3]{graphics/subsystem_example_1_1.jpg}
	\end{itemize}}{}
		\alt<2>{\begin{itemize}
				\item{Zeitentwicklung $U = \exp(-i H_{\mathrm{O}}t)$ von $\rho_{\mathrm{O}} = \ketup\braup$}
				\begin{itemize}
					\item{$U\ketup = \frac{1}{\sqrt{2}}\del{\ketup + \ketdown}$}
					\item{Entropie bleibt konstant}
				\end{itemize}
				\end{itemize}
				\begin{columns}
					\begin{column}{0.55\textwidth}
						\begin{beamerboxesrounded}{}
							\begin{align*}
							\rho^{\prime}_{\mathrm{O}} = U\rho_{\mathrm{O}}U^{\dagger}
							=&\frac{1}{2}(\ketup\braup + \ketup\bradown \\ &+ \ketdown\braup + \ketdown\bradown)
							\end{align*}
							\vspace{-0.5cm}
						\end{beamerboxesrounded}
					\end{column}
					\begin{column}{0.35\textwidth}
						 \centering
						 \includegraphics[scale=0.35]{graphics/subsystem_example_1_2.jpg}
					\end{column}
				\end{columns}}{}
				
		\alt<3>{\begin{itemize}
				\item{$H_{\mathrm{OE}}$: Dekohärenz (vollständig)}
				\begin{itemize}
					\item{Entropie nimmt zu}
				\end{itemize}
				\vspace{0.2cm}
				%				\centering
				%				\includegraphics[scale=0.3]{graphics/subsystem_example_1_1.jpg}
			\end{itemize}
			\begin{columns}
				\begin{column}{0.55\textwidth}
					\begin{beamerboxesrounded}{}
						\begin{equation*}
						\rho^{\prime\prime}_{\mathrm{O}} = \frac{1}{2}(\ketup\braup + \ketdown\bradown)
						\end{equation*}
						\vspace{-0.5cm}
					\end{beamerboxesrounded}
				\end{column}
				\begin{column}{0.35\textwidth}
					\centering
					\includegraphics[scale=0.35]{graphics/subsystem_example_1_3.jpg}
				\end{column}
			\end{columns}}{}
		\alt<4>{\begin{itemize}
				\item{$H_{\mathrm{OS}}$: Messung von S an O}
				\begin{itemize}
					\item{$U = \exp(-i \int\!\!\!H_{\mathrm{OS}}\dif t)$, Messung schnell}
					\item{$U\ketneutreyup = \ketsmileyup$, $U\ketneutreydown = \ketfrownydown$}
					\item{$\rho_{\mathrm{OS}} = \ketneutrey\braneutrey \otimes \frac{1}{2}(\ketup\braup + \ketdown\bradown)$}
					\item{Entropie nimmt ab}
				\end{itemize}
				\vspace{0.2cm}
				%				\centering
				%				\includegraphics[scale=0.3]{graphics/subsystem_example_1_1.jpg}
			\end{itemize}
			\begin{columns}
				\begin{column}{0.55\textwidth}
					\begin{beamerboxesrounded}{}
						\begin{empheq}{align*}
						\rho^{\prime}_{\mathrm{OS}} = U\rho_{\mathrm{OS}}U^{\dagger} = &\frac{1}{2}(\ketsmileyup\brasmileyup\\ &+ \ketfrownydown\brafrownydown)
						\end{empheq}
						\vspace{-0.5cm}
					\end{beamerboxesrounded}
				\end{column}
				\begin{column}{0.35\textwidth}
					\centering
					\includegraphics[scale=0.35]{graphics/subsystem_example_1_4.jpg}
				\end{column}
			\end{columns}}{}
%			\alt<5>{\begin{columns}
%					\begin{column}{0.45\textwidth}
%						\centering
%						\includegraphics[scale=0.35]{graphics/subsystem_example_1_4.jpg}
%					\end{column}
%					\begin{column}{0.45\textwidth}
%						\centering
%						\includegraphics[scale=0.35]{graphics/subsystem_example_1_4.jpg}
%					\end{column}
%				\end{columns}}{}}{}		
\end{frame}
		\begin{frame}{Beispiel: $H_{\mathrm{S}}$, $H_{\mathrm{SE}}$}
	\alt<1,2>{\begin{itemize}
		\item{Zeitentwicklung von S}
		\begin{itemize}
			\item{$U = \exp(-i \int\!\!\!H_{\mathrm{OS}}\dif t)$, Zeit kurz}
			\item{$U\ketneutrey = \frac{1}{\sqrt{2}}\del{\ketsmiley + \ketfrowny}$, $\rho_{\mathrm{S}} = \ketneutrey\braneutrey$}
		\end{itemize}
	\end{itemize}
				\begin{columns}
					 \alt<1>{\begin{column}{\textwidth}
					 	\vspace{0.25cm}
		 			 	\centering
		 			 	\includegraphics[scale=0.35]{graphics/subsystem_example_1_1.jpg}
					 	\end{column}}{}
					\alt<2>{\begin{column}{0.55\textwidth}
						\begin{beamerboxesrounded}{}
							\begin{align*}
							\rho^{\prime}_{\mathrm{S}} = U\rho_{\mathrm{S}}U^{\dagger}
							=&\frac{1}{2}(\ketsmiley\brasmiley + \ketsmiley\brafrowny \\ &+ \ketfrowny\brasmiley + \ketfrowny\brafrowny)
							\end{align*}
							\vspace{-0.5cm}
						\end{beamerboxesrounded}
					\end{column}
					\begin{column}{0.35\textwidth}
						 \centering
						 \includegraphics[scale=0.35]{graphics/subsystem_example_2_2.jpg}
					\end{column}}{}
				\end{columns}}{}	
	\alt<3>{\begin{itemize}
			\item{$H_{\mathrm{SE}}$: Dekohärenz des Subjekts}
		\end{itemize}
		\begin{columns}
			\begin{column}{0.55\textwidth}
				\begin{beamerboxesrounded}{}
					\begin{equation*}
					\rho^{\prime\prime}_{\mathrm{S}} = \frac{1}{2}(\ketsmiley\brasmiley + \ketfrowny\brafrowny)
					\end{equation*}
					\vspace{-0.5cm}
				\end{beamerboxesrounded}
			\end{column}
			\begin{column}{0.35\textwidth}
				\centering
				\includegraphics[scale=0.35]{graphics/subsystem_example_2_3.jpg}
			\end{column}
		\end{columns}
	\begin{itemize}
		\item{Auf welchen Zeitskalen läuft Dekohärenz im Gehirn ab?}
	\end{itemize}}{}			
		
\end{frame}
		
		
		
	\section{Dekohärenz von Gehirnprozessen}
		\separatorslide
		\begin{frame}{Superposition von Neuronen}
\begin{columns}
		\alt<1>{\begin{column}{0.55\textwidth}
			\begin{itemize}
				\item{Neuronen: Bausteine des menschlichen Gehirns \textasciitilde $10^{11}$}
				\begin{itemize}
					\item{komplexes Netzwerk}
					\item{Verbindung mit dem Bewusstsein anzunehmen}
				\end{itemize}
				\item{Zwei mögliche Zustände}	
				\begin{itemize}
					\item{feuern $\leftrightarrow$ nicht feuern}
				\end{itemize}
			\end{itemize}
		\end{column}
		\begin{column}{0.4\textwidth}
			\centering
			\includegraphics[scale=0.8]{graphics/neuron.jpg}\,\cite{neurons}
		\end{column}}{}
		\alt<2>{\begin{column}{0.45\textwidth}
			\begin{itemize}
				\item{Einfache Annahmen}
				\begin{beamerboxesrounded}{Anzahl der Na$^{\textcolor{white}{+}}$-Ionen}
					\begin{empheq}{equation*}
						N = \frac{\pi dLf\epsilon_{0}}{qh}\del{U_{1} -U_{0}}
					\end{empheq}
					\vspace{-0.5cm}
				\end{beamerboxesrounded}	

			\item{$h=\SI{8}{nm}$, $d=\SI{10}{\micro\metre}$,\\ $L=\SI{10}{\centi\metre}$, $f=\num{e-3}$,\\
				  $U_{0} = \SI{-0.07}{\volt}$, $U_{1} = \SI{0.03}{\volt}$}
				\begin{itemize}
					\item{$N \sim \num{e6}$}
				\end{itemize}
			\end{itemize}
		\end{column}
		\begin{column}{0.45\textwidth}
			\centering
			\includegraphics[scale=0.18]{graphics/neuron_schematic.jpg}\\\hfill\cite{Tegmark_99}
		\end{column}}{}
	\end{columns}	
	\alt<2>{\begin{itemize}
		\item[$\Rightarrow$]{Räumliche Superposition von \num{e6} Na$^{+}$-Ionen, mit Abstand  $\sim\mathcal{O}(\SI{10}{nm})$}
	\end{itemize}}{}
\end{frame}
		\begin{frame}{Dekohärenz von Neuronen}
	\begin{itemize}
		\item{Unterschiedliche Wechselwirkungen}
		\begin{itemize}
			\item{Stöße zwischen Na$^{+}$-Ionen, anderen Ionen und H$_{2}$O-Molekülen }
			\item{Coulombabstoßung der Na$^{+}$ von andern Ionen}
		\end{itemize}
		\item{Abschätzung der Größenordnung, unteres Limit}
		\item{Coulombabstoßung: nächstes Ion größter Beitrag}
		\item{Stoßprozesse dekohärieren Ion auf de-Broglie Wellenlänge des Stoßteilchens}
	\end{itemize}
\end{frame}
				
	\section{Bewusstsein als Aggregatzustand}
		\separatorslide
		\begin{frame}{Bewusstsein als Aggregatzustand}
	\begin{columns}
		\begin{column}{0.55\textwidth}
			\begin{itemize}
				\item{Aggregatzustände durch Eigenschaften unterscheidbar}
				\item{Ähnliche Konzepte bereits erdacht}
				\begin{itemize}
					\item{\emph{Computronium}}
				\end{itemize}
				\item{Welche Eigenschaften muss \emph{Perceptronium} besitzen?}
			\end{itemize}
		\end{column}
		\begin{column}{0.45\textwidth}
			\centering
			\includegraphics[scale=0.15]{graphics/states_of_matter.jpg}\\\hfill\cite{pic_stone,pic_water_droplet,steam_eruption,lightning_teslacoil}
		\end{column}
	\end{columns}
\end{frame}
		\begin{frame}{Eigenschaften von Perceptronium}
	\centering
	\only<1>{\includegraphics[scale=0.3]{graphics/property_principles.jpg}\,\cite{Tegmark_15_long}}
	\only<2>{\hspace{-0.08cm}\includegraphics[scale=0.3]{graphics/property_principles_1.jpg}\,\cite{Tegmark_15_long}}
%	\only<3>{\hspace{-0.16cm}\includegraphics[scale=0.3]{graphics/property_principles_2.jpg}\,\cite{Tegmark_15_long}}
	%\only<4>{\hspace{-0.24cm}\includegraphics[scale=0.3]{graphics/property_principles_3.jpg}\,\cite{Tegmark_15_long}}
	%\only<5>{\hspace{-0.32cm}\includegraphics[scale=0.3]{graphics/property_principles_4.jpg}\,\cite{Tegmark_15_long}}
\end{frame}
		\begin{frame}{Integrierte Information}
	\begin{itemize}
		\item{Aktive Forschung z.B. in der Neurowissenschaft}
		\begin{itemize}
			\item{G. Tononi (\textsc{Integrated Information Theory})\,\cite{Tononi_08}}
  		\end{itemize}
	\end{itemize}
	\begin{beamerboxesrounded}{Integrierte Information $\textcolor{white}{\Phi}$ (abgewandelt)}
		\begin{empheq}{equation*}
			\Phi = I_{\mathrm{min}} = \min_{\rho_1,\rho_2} \del{S(\rho_{1}) + S(\rho_{2}) - S(\rho)}
		\end{empheq}
		\vspace{-0.5cm}
		\begin{empheq}{equation*}
			\small I: \text{Transinformation}, S = - \Tr[\rho\log_{2}(\rho)]
		\end{empheq}
		\vspace{-0.5cm}
	\end{beamerboxesrounded}
	\begin{itemize}
		\item{Minimale Transinformation nach einem Schnitt der das System in zwei teilt}
			\begin{itemize}
				\item{\Quote{the cruelest cut} - Giulio Tononi}
				\item{Maximale Unabhängigkeit der Teilsysteme,\\ $\Phi = 0 \Leftrightarrow$ vollständig unabhängig}
			\end{itemize}
	\end{itemize}
\end{frame}
		\begin{frame}{Integrierte Information}
		\begin{columns}
			\alt<1,2>{\begin{column}{0.55\textwidth}
				\begin{itemize}
					\item{Betrachtbar als Speicherung von Information,
						mit Fehlerkorrektur-Mechanismus}
					\item{Integrierte Information für $k$ zufällig ausgewählte
						  14-bit-Folgen}
					\begin{itemize}
						\item{Maximum bei $k \approx 2^{7}$}
					\end{itemize}
				\end{itemize}
			\end{column}
			\begin{column}{0.5\textwidth}
				\centering
				\alt<1>{\includegraphics[scale=0.14]{graphics/qrcode_whole.jpg}}{}
				\alt<2>{\includegraphics[scale=0.14]{graphics/qrcode_damaged.jpg}}{}
			\end{column}}{}

		\alt<3>{\begin{column}{0.55\textwidth}
				\begin{itemize}
					\item{Physikalische Systeme}
					\begin{itemize}
						\item{\enquote{Eierkarton}-Potential ($16\times16$)}
						\item{Oben 256 Minima, $S(\text{Grundzustand}) = 8$}
						\item{Unten 16 Minima, $S(\text{Grundzustand}) = 4$}
					\end{itemize}
					\item{Ort $\del{x,y}$ als zwei 4-bit Zahlen: $0_{2} \text{ -- } 15_{2}$}
					\item{Integration}
						\begin{itemize}
							\item{Oben schlecht $\Phi = 0$}
							\item{Unten gut $\Phi = 2$}
						\end{itemize}
				\end{itemize}
			\end{column}
			\begin{column}{0.5\textwidth}
				\centering
				\includegraphics[scale=0.18]{graphics/egg_crate_potentials.jpg}\,\cite{Tegmark_15_long}
			\end{column}}{}	
		\end{columns}
			\centering
			\alt<1,2>{\includegraphics[scale=0.16]{graphics/integrated_information_graph.jpg}\cite{Tegmark_15_long}}{}
\end{frame}
		\begin{frame}{Probleme mit Integration}
	\begin{itemize}
		\item{Informationsgehalt des Zustand $\rho$ und des Systems $H$}
	\end{itemize}
		\begin{beamerboxesrounded}{\enquote{Eierkarton} -Potential mit $\textcolor{white}{k}$ Minima}
			\begin{empheq}{equation*}
			S(\rho) \sim \log_{2}(\text{\# möglicher Zustände}) \sim n
			\end{empheq}
			\vspace{-0.5cm}
			\begin{empheq}{equation*}
			S(H) \sim \log_{2}(\text{\# möglicher } H) \sim kn
			\end{empheq}
			\vspace{-0.5cm}
			\end{beamerboxesrounded}
	\begin{itemize}
		\item{Gehirn mit \num{e11} Neuronen}
	\end{itemize}
				\begin{beamerboxesrounded}{Maximale Integration}
					\begin{empheq}{equation*}
					S(H) \sim \sqrt{2^{n}}\frac{n}{2} \sim 10^{10^{10}} \text{bit}
					\end{empheq}
					\vspace{-0.5cm}
				\end{beamerboxesrounded}
		\begin{itemize}
			\item{Notwendige Dynamik viel zu komplex}
		\end{itemize}
\end{frame}
%		\subsection{Unabhängigkeit}
%		\separatorslide
		\input{contents/properties}
		% !TeX root = ../paper_observer_consciousness.tex

\begin{wrapfigure}{r}{0.4\textwidth}
	\centering
	\includegraphics[scale=0.8]{graphics/presentation_qm_extra.pdf}
	\caption{Darstellung der Transinformation $I$ eines 2-bit/2-qbit-Systems in Abhängigkeit seiner
		Entropie $S$. Klassische Systeme können nur in dem hellen Dreieck liegen, während quantenmechanisch
		zusätzlich das dunklen Dreieck möglich ist. Markiert sind die 4 Zustände:
		(1): $\rho = \frac{1}{2} \del{\ketup\!\braup + \ketdown\!\bradown}$, (2): $\rho = \ketup\!\braup$,
		(3): $\rho = \frac{1}{4} (\ketup\!\braup + \ketdown\!\braup+ \ketup\!\bradown + \ketdown\!\bradown)$,
		(4): $\rho = \frac{1}{2} \del{\ketup\!\ketup + \ketdown\!\ketdown}$. Der hell graue Punkt stellt für dieses 
		2-qbit-System die maximal mögliche Integrierte Information dar. \label{fig:independence_plot}}
\end{wrapfigure}  

Wie im vorherigen Abschnitt gesehen, führt maximale Integration zu einer so komplexen Dynamik,
dass diese im menschlichen Gehirn nicht realisiert sein kann. Im Folgenden soll nun  
betrachtet werden wie groß die Unabhängigkeit zwischen Teilsystemen werden kann und welche Konsequenzen
aus dieser Unabhängigkeit resultieren. Als Maß kann dabei wiederum die Integrierte Information dienen, für maximal 
unabhängige Teilsysteme hat diese den Wert $\Phi = 0$.

Bei der Betrachtung von Unabhängigkeit stellt man eine großen Unterschied zwischen klassischen und 
quantenmechanischen Systemen fest. Zur Verdeutlichung sollen hier ein klassisches 2-bit- und ein quantenmechanisches
2-qbit-System betrachtet werde. Die Zustände, die von diesen beiden Systemen eingenommen werden 
können alle als Punkt in dem Koordinatensystem aus $(S,I)$ in \cref{fig:independence_plot} eingetragen werden.
Dabei ist das klassische System auf das helle Dreieck beschränkt, während das 2-qbit-System zusätzlich im dunklen 
Dreieck liegen kann.



Die Berechnung der Integrierten Information des klassischen Systems erweist sich als trivial, da nur ein 
einziger Schnitt in zwei Teilsysteme (jeweils ein bit) möglich ist und somit $\Phi = I$ gilt.
Für das Quantensystem ergibt sich jedoch das Problem, dass jede Basis des Hilbertraums gleich bedeutend
ist und daher die Minimierung der Transinformation kontinuierlich über alle unitären Transformationen 
durch geführt werden muss.
\begin{empheq}{equation}
	\Phi = \displaystyle\min_{U} I\del{U\rho U^{\dagger}}
\end{empheq}  
Dies führt dazu das selbst der Zustand (4) in \cref{fig:independence_plot}, einer der vier maximal verschränkten 
Bell-Zustände, in einen reinen Zustand transformiert werden kann, sodass für diesen $\Phi= 0$ gilt.
Die für Quantenzustände maximal mögliche Integrierte Information kann für ein System der betrachteten
Größe zu $\Phi = \num{0.2516}$ bit bestimmt werden. Dieser Wert ist als grauer Punkt in \cref{fig:independence_plot} 
eingezeichnet. Und auch für größere System zeigt sich, dass
$\Phi$ für klassische Systeme annähernd linear mit der Systemgröße steigt während $\Phi$ für Quantensysteme 
gegen null geht. Damit zeigt sich das Zustände von Quantensystemen im Vergleich zu klassischen Zuständen wesentlich 
unabhängiger sind.

Betrachtet man wiederum die Integrierte Information des gesamten Systems, beschrieben durch $H$, 
der allgemein in drei Teile separiert werden kann
\begin{empheq}{equation}
	H = H_{1} \otimes I  +  I \otimes H_{2} + H_{3}
\end{empheq}
so ist leicht zu verstehen, dass diese am geringsten ist, wenn der Wechselwirkungsterm $H_3$ minimal wird.
Wird eine solche Separation gefunden gilt, da $H_1$ und $H_2$ näherungsweise unabhängig sind, dass 
$\comm{H_i}{H_j} = 0$ und $\comm{H_i}{H_3} = 0$ für $i,j \in {1,2}$. Das Vertauschen des System-Hamiltonoperators
mit dem Wechselwirkungsterm führt nun aber dazu, dass Zustände durch Dekohärenz in einen zeitunabhängigen
Zustand getrieben werden. Trennt man das Universum also in möglichst unabhängige Teile auf, so kommt jegliche 
Veränderung zum erliegen.     



		%TODO: Wort auswählen
\begin{frame}{Unabhängigkeit von Quantenzuständen}
	\begin{itemize}
		\item{Bell-Paar und perfekt korrelierte klassische bits,
			  keine integrierte Information}
		\begin{itemize}
			\item{Wie hoch kann $\Phi$ quantenmechanisch werden?}
		\end{itemize}
	\end{itemize}
	\begin{beamerboxesrounded}{$\textcolor{white}{\rho}$-Diagonalitäts-Satz}
		Die Transinformation $I$ ist in einer Basis am geringsten,\\ in der $\rho$
		diagonal ist.
	\end{beamerboxesrounded}
		\begin{itemize}
			\item{für $\rho \in \mathbb{R}^{4\times4}$ mit $n$ nicht entarteten Eigenwerten,\\
				Diskrete Minimierung über $n!$ Permutationen}
			\begin{itemize}
				\item{höhste Werte für $\Phi$ bei $\rho \propto \rho^{2}$,
					  $k$ Eigenwerte sind $k^{-1}$ übrige sind 0}
				\item{$n=4, k=3 \Rightarrow \Phi \approx \num{0.2516}$ bit}
			\end{itemize}
			\item{klassische: $\Phi = \mathcal{O}(n)$,\\ quantenmechanisch: $n \to \infty \Rightarrow \Phi \to 0$  }
		\end{itemize}
\end{frame}
		\begin{frame}{Unabhängigkeit von Quantensytemen}
	\alt<1>{\begin{itemize}
		\item{Allgemeiner Hamiltonoperator $H$}
	\end{itemize}
	\begin{beamerboxesrounded}{Separation in Teilsysteme und Wechselwirkung}
		\begin{empheq}{equation*}
			H = H_{1} \otimes I  +  I \otimes H_{2} + H_{3}
		\end{empheq}
		\vspace{-0.5cm}
	\end{beamerboxesrounded}
		\begin{itemize}
			\item{$H_{3} = 0$: $H_{1},H_{2}$ \enquote{parallele Universen}}
					\begin{itemize}
						\item{der härteste Schnitt bei minimalem $H_{3}$}
					\end{itemize}
		\end{itemize}}{}	
	\alt<1,2>{\begin{beamerboxesrounded}{$\textcolor{white}{H}$-Diagonalitäts-Satz}
			Der Hamiltonoperator $H$ ist immer in der Energieeigenbasis, in der $H$ diagonal ist maximal separierbar 
			( $\norm{H_3}$ minimal).
	\end{beamerboxesrounded}}{}
	\alt<2>{\begin{itemize}
			\item{In dieser Separation kommutieren $H_1$ und $H_2$ mit einander und mit $H_3$}
			\begin{itemize}
				\item{ähnlich zur Zeiger-Basis von Zurek, Zustände kommutieren mit 
					WW-Hamiltonoperator\,\cite{Zurek_01}}
			\end{itemize}
		\end{itemize}}{}
				\alt<3>{\begin{columns}
					\begin{column}{.55\textwidth}
					\begin{itemize}
					\item{Dekohärenz treibt System $H_1$ mit $\comm{H_1}{H_3} = 0$ in einen zeitunabhängigen Zustand}
					\begin{itemize}
						\item{Reiner Zustand $\longrightarrow$ vollständig gemischter Zustand}
					\end{itemize}
					\end{itemize}	
					\end{column}
					\begin{column}{.45\textwidth}
						\centering
						\includegraphics[scale=0.25]{graphics/blochsphere_independance.jpg}\,\cite{Tegmark_15_long}
					\end{column}
				\end{columns}}{}
				
		\alt<3>{\begin{beamerboxesrounded}{Unabhängigkeits-Paradoxon}
				Zerlegt man das Universum in maximal unabhängige Objekte,
				kommt jegliche Veränderung zum erliegen.
			\end{beamerboxesrounded}}{}
\end{frame}
		%\begin{frame}{Autonomie}
	\alt<1>{\begin{itemize}
		\item{Vereinigung von \textcolor{vertexLightGrey}{Dynamik} und Unabhängigkeit}
			\begin{itemize}
				\item{\textcolor{vertexLightGrey}{Informationsverarbeitung}}
			\end{itemize}
		\item{Maß für Dynamik}
	\end{itemize}
	\begin{beamerboxesrounded}{Energie-Kohärenz}
		\begin{empheq}{equation*}
			\delta H = \frac{1}{\sqrt{2}} \norm{\dot{\rho}} = \sqrt{\Tr\sbr{H^{2}\rho^{2} - H\rho H\rho}}
		\end{empheq}
		\vspace{-0.5cm}
	\end{beamerboxesrounded}}{}
	\alt<2>{\begin{itemize}
		\item{Unterschiedliche Grade an Dynamik:}
	\end{itemize}
	\begin{columns}
		\begin{column}{0.3\textwidth}
			\centering
			\includegraphics[scale=.3]{graphics/autonomy_static.jpg}
		\end{column}
		\begin{column}{0.3\textwidth}
			\centering
			\includegraphics[scale=.3]{graphics/autonomy_chaotic.jpg}
		\end{column}
		\begin{column}{0.3\textwidth}
			\centering
			\includegraphics[scale=.3]{graphics/autonomy_simple.jpg}
		\end{column}
	\end{columns}
	\begin{itemize}
		\item{Reduktion der maximalen Energie-Kohärenz um wenige Prozent}
			\begin{itemize}
				\item{Komplexe, chaotische Dynamik möglich}
			\end{itemize}
	\end{itemize}}{}
\end{frame}
		
		
		
		
	
	\section{Zusammenfassung}
		\separatorslide
	\begin{frame}{Zusammenfassung}
	\begin{itemize}
		\item{Schon allgemein \enquote{schwierige Probleme} bei
			  Beschäftigung mit dem Bewusstsein}
		\item{In der Physik nicht anders, schon Beobachter unklar}
			\begin{itemize}
				\item{Schon allgemein Einbeziehung des Bewusstsein notwendig;
					  Many-Minds-Interpretation}
			\end{itemize}
		\item{Beobachter als Teilsystem betrachtet}
			\begin{itemize}
				\item{Hohe Korrelation zu kleiner Auswahl an Eigenschaften der Umwelt}
			\end{itemize}
		\item{Dekohärenz-Zeiten für Neuronen-Superpositionen}
					\begin{itemize}
						\item{klassische Beschreibung notwendig}
					\end{itemize}
		\item{Integrierte Information und Autonomie notwendige Eigenschaften von
			  bewusster Materie}
	\end{itemize}
\end{frame}
	
	\section*{Backup}
	\separatorslide	
	\begin{frame}{Autonomie}
		\only<1>{\hspace{-0.24cm}\includegraphics[scale=0.3]{graphics/property_principles_3.jpg}\,\cite{Tegmark_15_long}}
		\only<2>{\hspace{-0.32cm}\includegraphics[scale=0.3]{graphics/property_principles_4.jpg}\,\cite{Tegmark_15_long}}
		\alt<3>{\begin{itemize}
				\item{Vereinigung von \textcolor{vertexLightGrey}{Dynamik} und Unabhängigkeit}
				\begin{itemize}
					\item{\textcolor{vertexLightGrey}{Informationsverarbeitung}}
				\end{itemize}
				\item{Maß für Dynamik}
			\end{itemize}
			\begin{beamerboxesrounded}{Energie-Kohärenz}
				\begin{empheq}{equation*}
					\delta H = \frac{1}{\sqrt{2}} \norm{\dot{\rho}} = \sqrt{\Tr\sbr{H^{2}\rho^{2} - H\rho H\rho}}
				\end{empheq}
				\vspace{-0.5cm}
			\end{beamerboxesrounded}}{}
			\alt<4>{\begin{itemize}
					\item{Unterschiedliche Grade an Dynamik:}
				\end{itemize}
				\begin{columns}
					\begin{column}{0.3\textwidth}
						\centering
						\includegraphics[scale=.3]{graphics/autonomy_static.jpg}
					\end{column}
					\begin{column}{0.3\textwidth}
						\centering
						\includegraphics[scale=.3]{graphics/autonomy_chaotic.jpg}
					\end{column}
					\begin{column}{0.3\textwidth}
						\centering
						\includegraphics[scale=.3]{graphics/autonomy_simple.jpg}\,\cite{Tegmark_15_long}
					\end{column}
				\end{columns}
				\begin{itemize}
					\item{Reduktion der maximalen Energie-Kohärenz um wenige Prozent}
					\begin{itemize}
						\item{Komplexe, chaotische Dynamik möglich}
					\end{itemize}
				\end{itemize}}{}
\end{frame}
	
	\section*{Quellen}
	\separatorslide
	
	%\nocite{GRS,INES_Manual,bmud,BFS,kernenergieDE,wikiRS}
	\nocite{Zeh_00}
	\begin{frame}[allowframebreaks]{Quellen}
		\printbibliography
	\end{frame}

\end{document}